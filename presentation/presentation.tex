\documentclass[xcolor=dvipsnames]{beamer}
% \usetheme{CambridgeUS}
% \useinnertheme{rectangles}
% \useoutertheme{infolines}
% \usecolortheme[named=Brown]{structure}
\definecolor{UniBlue}{RGB}{100,80,200}
\setbeamercolor{title}{fg=UniBlue}
\setbeamercolor{frametitle}{fg=UniBlue}
\setbeamercolor{structure}{fg=UniBlue}
\usepackage[utf8]{inputenc}
\usepackage{minted}
\usepackage{tikz}
\usepackage{graphicx}
\usepackage{caption}
\usepackage{subcaption}
\usepackage[tt=false]{libertine}
\def\checkmark{\tikz\fill[scale=0.4](0,.35) -- (.25,0) -- (1,.7) -- (.25,.15) -- cycle;} 
\def\CC{{C\nolinebreak[4]\hspace{-.05em}\raisebox{.4ex}{\tiny\bf ++}}}


\title{\textsc{Citadel}: A Trusted Reference Monitor for Linux using Intel SGX Enclaves}
\author{A.H.~Bell-Thomas}
\institute{Computer Laboratory, University of Cambridge}
\date{\scriptsize $26^{\text{th}}$ June, 2020}

\begin{document}

\frame{\titlepage}

\begin{frame}{Background}
\pause
{\large
\begin{enumerate}
    \item \textbf{Reference Monitor} \\
    \pause
    $\;\;\rightsquigarrow$ \textit{Information Flow Control}
    \pause
    \vspace{1cm}
    \item \textbf{Intel SGX}
\end{enumerate}
}
\end{frame}

\begin{frame}{Information Flow Control}
    
\begin{itemize}
    \item Different
\end{itemize}
    
\end{frame}

\begin{frame}{Intel SGX}
    
\end{frame}


\begin{frame}{Motivation}
    
\end{frame}

\begin{frame}{\textsc{Citadel}}
    
\end{frame}

\begin{frame}{Results}
    
\end{frame}

\begin{frame}{Related Works}
    
\end{frame}


\end{document}

