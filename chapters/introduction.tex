


\paragraph{} The task of defending computer systems against malicious programs and affording isolation to protected system components has always been exceedingly challenging to achieve. A system's \textit{Trusted Computing Base}, or \textit{TCB}, defines the minimal set of software, firmware, and hardware components critical to establish and maintain system security and integrity. This traditionally includes, amongst others; the OS kernel; device drivers; device firmware; and the hardware itself. Compromise of a trusted component inside a system's \textit{TCB} is a direct threat to any secure application running on it. A common approach to hardening a system's security is to minimise its \textit{TCB}, diminishing its potential \textit{attack surface}. 

\paragraph{} A modern trend is to outsource the physical layer of a system to a foreign party, for example a \textit{cloud provider} --- this is beneficial both in terms of cost and flexibility, but confuses many security considerations which assume that the physical layer itself can be trusted. In this context there is no guarantee of this, as the physical layer is usually provided as a \textit{virtual machine}, inflating the system's \textit{TCB} with an external and transparent software layer, the underlying \textit{hypervisor}.

\paragraph{} The concept of \textit{Trusted Execution Environments}, \textit{TEEs}, has been explored by the security community for a very long time as potential protection against this, providing isolated processing contexts in which an operation can be securely executed irrespective of the rest of the system --- one such example is software \textit{enclaves}. \textit{Enclaves} are general-purpose \textit{TEEs} provided by a CPU itself, protecting the logic found inside at the architectural level. Intel's Software Guard Extensions (SGX) is the most prolific example of a \textit{TEE}, affording a \textit{black-box} environment and runtime for arbitrary apps to execute under. Introduced by Intel's \textit{Skylake} architecture, a partial view of the platform's working can be found in whitepapers and previous publications.

\paragraph{} An alternative approach to policing components in a system is via the use of \textit{Information Flow Control} (IFC). Enforced using a \textit{reference monitor}, IFC models how and where data is allowed to move and be manipulated by a system at a granular level.

\paragraph{} The aim of this work is to explore methods of hardening Linux with an SGX-driven \textit{reference monitor} to track and protect host OS system resources using IFC methods. Further, it aims to reason what the future relationship between an OS and the enclaves it hosts, and whether complete isolation between them is the natural answer in a number of common situations.


\paragraph{Our Contributions}
\begin{itemize}
    \item A prototype implementation of a modular \textit{reference monitor} protected using Intel SGX, empowering \textit{information flow control} techniques to operate with autonomy and protection from the host operating system. Enforcement is achieved using a new \textit{Linux Security Module} embedded in the Linux kernel, with an overall \textit{TCB} of only a minimal footprint of the kernel alongside the enclave application.
    \item A userspace interposition library to near-transparently integrate unmodified applications to fully function under the new restrictions.
    \item A full port of the \textit{libtomcrypt} cryptography library for use inside an SGX enclave.
    \item A rigorous investigation of the performance implications of this approach, featuring a lightly-modified version of the \textit{Nginx} production webserver. Worst-case performance shows a $35$\% decrease in request throughput, with the common case reporting $7-11$\%. Additionally we report a median overhead of $39\,\mu s$ (IQR $26-72\,\mu s$, $n = 10^6$) per affected \textit{system call}, matching or surpassing similar, non-enclave-based, systems.
\end{itemize}