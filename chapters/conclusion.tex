\paragraph{} This dissertation presented \textsc{Citadel}, a modular, enclave-backed reference monitor to securely and verifiably implement \acrshort{ifc} methods in the Linux kernel. By separating policy decisions and enforcement, we demonstrated a feasible approach to deep kernel integration using Intel \acrshort{sgx}; the prototype leverages Linux's security framework to realise decisions at the lowest level of the \acrshort{os}. \textsc{Citadel} optimises for performance via an auxiliary library, which conservatively predicts a process's security context, enabling unobtrusive application integration.

\paragraph{} A full implementation of the \textsc{Nginx} webserver running on \textsc{Citadel} validates this work using real-world performance benchmarks; the most punitive trial produced a 25\% overhead, but other scenarios reported performance parity with the native Linux kernel. Verifying the methods presented here should be the next step, but an extension of \textsc{Citadel} in a distributed setting also has great potential; inter-machine attestation will likely establish an exceptional degree of trust between remote components.


\paragraph{} There is a way to go before \textsc{Citadel} is fully realised and production-ready, but this project successfully demonstrates the viability and potential of a symbiotic enclave-kernel relationship, which, in the long run, may prove valuable for both.